\documentclass[main]{subfiles}
\title{Intro}
\author{Chuwen Zhang}
\date{\today}
\begin{document}
\maketitle
{
    \setcounter{tocdepth}{3}
    \tableofcontents
}

\section{Preface}

This notes is written for basic knowledge on Interior Point Method (IPM). Also see \cite{ye_interior_1997}, \cite{roos_interior_2005}.
We begin with Linear Programming (LP), then move onto advanced problem classes.

\section{Interior Point Method for Linear Programming}

\subsection{Elements}

We inspect primal dual form of linear programming or linear optimization problems.

\begin{equation}
    \begin{aligned}
        \mn & c^T x         \\
        \st & A x=b, x\ge 0
    \end{aligned}
\end{equation}

\begin{equation}
    \begin{aligned}
        \mx & b^Ty                  \\
        \st & A^{T} y+s=c, s \geq 0
    \end{aligned}
\end{equation}

and the following remarks,

\begin{itemize}
    \item Feasible region \(\mathscr F_P, \mathscr F_D\) for primal dual respectively.
          We let \(F^\circ_{(\cdot)}\) be the relative interior.
          Strong duality holds if \(F^\circ_{P}, F^\circ_{P}\) are nonempty.
    \item
\end{itemize}

\subsubsection{The analytic center}

The \textbf{analytic center, AC} is defined as the minimizer of \textbf{potential function} on a non-empty set \(\Omega\).

If \(\Omega\) is defined (as usual) in form \(\Omega = \{x :  g(x) \ge 0\}\), assume \(\Omega^\circ \neq \emptyset\), the potential function / barrier is,

\begin{equation}
    \psi(x) = - \log g^Te
\end{equation}

% \begin{example} 
% \(\psi\) on positive orthant For the feasible set \(\Omega_x = \{x \ge 0\}\)
% \begin{equation}
%     \begin{aligned}
%         \psi(x)                                    & =  - \log x^Te    \\
%         \partial \psi / \partial x = 0 \Rightarrow & - \frac{e}{x} = 0 \\
%         \textrm{Alternatively}                     &                   \\
%                                                    & x \circ v = e     \\
%                                                    & v = 0
%     \end{aligned}
% \end{equation}
% \end{example}


\begin{equation}
    x \circ v = e, \quad Ax = b
\end{equation}
\begin{example}
    \(\psi\) on half space.  For the feasible set \(\Omega_y = \{y: c - A^Ty \ge 0\}\)
    \begin{equation}\label{lp.ac.example.y}
        \begin{aligned}
            \psi(y)                                    & = -  \log s ^T e        \\
            \partial \psi / \partial y = 0 \Rightarrow & - \frac{A}{c- A^Ty} = 0 \\
            \textrm{Alternatively}                     &                         \\
                                                       & c- A^Ty = s             \\
                                                       & s \circ u = e           \\
                                                       & Au = 0
        \end{aligned}
    \end{equation}
\end{example}
For the case with extra equalities (always active) \(\Omega_y = \{y: c - A^Ty \ge 0\} \bigcap \{y: Ay = b\}\), since the analytic center also lies in the original space, we have \(y^a, s^a\) satisfies \eqref{lp.ac.example.y} + \(\{y: Ay = b\}\).

\subsubsection{The central path}

\paragraph{Primal} notice the primal potential function and the barrier problem, \(\forall \mu \ge 0\)
\begin{equation}
    \begin{aligned}
        \mn & \psi(x) = c^Tx - \mu\log x^Te \\
        \st & Ax = b, x > 0
    \end{aligned}
\end{equation}

The first-order condition,
\begin{equation} \label{lp.central.path}
    \begin{aligned}
        x \circ s & =\mu e \\
        A x       & =b     \\
        A^T y+s   & =c
    \end{aligned}
\end{equation}

depicts the curvature parameterized by \(\mu\). The curvature is named as ``the central path''.
\begin{itemize}
    \item \(\mu \to 0\), \(x(\mu)\) solves primal problem that approaches to AC of the optimal face: \(\{x: c^Tx = z^\star, Ax=b, x\ge 0\}\)
    \item \(\mu \to \infty\), \(x(\mu)\) approaches to the analytic center (if bounded)
\end{itemize}

Now it's clear to see the central path is the ``path'' connecting the AC and optimal solution (if exists).

\paragraph{Dual} similarly, we can define the (dual) central path by the log-barrier function,
\begin{equation}
    \begin{aligned}
        \mn & \psi(x)                                 = b^Ty + \mu\log s^Te \\
        \st & c - A^Ty = s, s > 0
    \end{aligned}
\end{equation}
as well as the \(1^\textrm{st}\) order condition that coincides with \eqref{lp.central.path}.
\paragraph{Primal-Dual} A combining view on central path is to see \eqref{lp.central.path} as the parameterization simultaneously on \((x, y, s)\).
\begin{itemize}
    \item One see duality gap: \(c^Tx - b^Ty = n\mu\)
    \item \(x \circ s =\mu e\) is sometimes referred to as perturbed complementary condition (\textbf{TCC}).
\end{itemize}

\subsubsection{An overview of IPM}

Now it becomes clear to us, the interior point method gradually shrinks \(\mu \to 0\) hoping to approach the original problem.
\begin{itemize}
    \item The outer loop produces the central path \(\left(x(\mu), y(\mu), s(\mu)\right)\) (or a subset of interest)
    \item The inner subproblem solves the central path system \eqref{lp.central.path}
\end{itemize}

The Newton's equation for \eqref{lp.central.path} can be derived as,

\begin{equation}
    
\end{equation}

% IPM can be classified into several basic classes.
% \begin{itemize}
%     \item Karmarkar's Algorithm, \cite{karmarkar_new_1984}
%     \item Potential Reduction algorithm (\textbf{PR})
%     \item Path Following algorithm (\textbf{PF})
%     \item Affine Scaling algorithm (\textbf{AS})
% \end{itemize}

\subsection{Timeline}

\bibliography{headers/ipm}
\bibliographystyle{headers/spmpsci}
\addcontentsline{toc}{section}{References}
\end{document}
